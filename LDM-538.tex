\documentclass[DM,lsstdraft,STS,toc]{lsstdoc}

% lsstdoc documentation: https://lsst-texmf.lsst.io/lsstdoc.html

% Package imports go here.
\usepackage{enumitem}
\input meta.tex
% Local commands go here.

% To add a short-form title:
% \title[Short title]{Title}
\title{Data Management L1 Enclaves Test Specification}

% Optional subtitle
% \setDocSubtitle{A subtitle}

\author{%
Michelle Butler, Jim Parsons, Michelle Gower 
}

\setDocRef{LDM-538}

\date{\today}

% Optional: name of the document's curator
% \setDocCurator{Michelle Butler}

\setDocAbstract{%
This is the test specifications for LDM-503-3 and LDM503-4b.   These are the tests of the trigger of the camera to produce a fits file with proper headers, and then taking one of those images and placing it into the data back bone (DBB).  
}

% Change history defined here.
% Order: oldest first.
% Fields: VERSION, DATE, DESCRIPTION, OWNER NAME.
% See LPM-51 for version number policy.
\setDocChangeRecord{%
  \addtohist{1}{2018-04-18}{initial release - draft version}{Michelle Butler }
}

\begin{document}
\def\product{DM L1 Enclaves}
\setDocCompact{true}


\title[Test Spec for \product]{\product~Test Specification}

\setDocRef{\lsstDocType-\lsstDocNum}
\setDocDate{\vcsdate}

\setDocAbstract {
This document describes the detailed test specification for the \product{}.
}

\setDocUpstreamLocation{\url{https://github.com/lsst/ldm-538}}
\setDocUpstreamVersion{\vcsrevision}

\maketitle

\section{Introduction}
\label{sec:intro}

This document specifies the test procedure for the \product{}.

The \product{} is the component of the LSST system which is triggering the camera which then leads to: 
\begin{itemize}

  \item{Gathering the data; }
  \item{Storing the data in a collection location;}
  \item{Notifing the Data Back Bone (DBB) that a file has been created;}
  \item{DBB stores the files including the providence and the MD5 sums for future reference}

\end{itemize}


\subsection{Objectives}
\label{sec:objectives}


This document builds on the description of LSST Data Management's approach to
testing as described in \citeds{LDM-503} to describe the detailed tests that
will be performed on the \product{} as part of the verification of the DM system.

It identifies test designs, test cases and procedures for the tests, and the
pass/fail criteria for each test.

\subsection{Scope}
\label{sec:scope}

This document describes the test procedures for the following components of
the LSST system (as described in \citeds{LDM-148}):

\begin{itemize}

  \item{L1 DAQ trigger and write out FITS files with providence  }

  \item{DBB retrieve from from L! and ingest into filesystem }

\end{itemize}

\subsection{Applicable Documents}
\label{sec:docs}

\addtocounter{table}{-1}

\begin{tabular}[htb]{l l}
\citeds{LDM-148} & LSST DM System Architecture \\
\citeds{LSE-230} & ??  \\

\end{tabular}
LSE-230, LSE-72, LSE-73, LSE-68, and LSE-70. 
LSE-230, LSE-72, LSE-73, LSE-68, and LSE-70

\subsection{References\label{sect:references}}
\renewcommand{\refname}{}
\bibliography{lsst,refs,books,refs_ads}

%\subsection{Definitions, acronyms, and abbreviations \label{sect:acronyms}} % include acronyms.tex generated by the acronyms.csh (GaiaTools)
%\input{acronyms}


%----------------------------------------------------
% TASK IDENTIFICATION - APPROACH
%----------------------------------------------------
\section{Approach}
\label{sec:approach}

The major activities to be performed are to:

\begin{itemize}

  \item{Spectrograph Image Data Fetch from DAQ. Integration of Image Data with header metadata from the DM Header Service (DMHS), producing FITS files. This test is a lead up test to a Pathfinder Activity to be held on Feb. 26th, 2018 referred to as the Daq-to-DM Files exercise. It will also use the OCS and CCS as well as the DMHS and the DM. It will insures that DAQ data can be fetched and properly integrated with header service output when all of the vendors are involved.}

 
\end{itemize}

\subsection{Tasks and criteria}
\label{sec:tasks}

The following are the major items under test:

\begin{itemize}

  \item{Daq-to-DM Files.  It will insures that DAQ data can be fetched and properly integrated with header service output ;}

 

\end{itemize}

\subsection{Features to be tested}
\label{sec:feat2test}

\begin{itemize}

  \item{Header service and l1 ATS device will be started and state to enable;}
  \item{Header service and L1 DM system respond with correct messages;}
  \item{Take images and deposit well formed FITS files at a file landing facility within the L1 Archive Controller component.   Prices locations is configurable within the system wide L1SystemCfg.yaml file.}

\end{itemize}


\subsection{Pass/fail criteria}
\label{sec:passfail}

The results of all tests will be assessed using the criteria described in
\citeds{LDM-503} \S4.

Note that, when executing pipelines, tasks or individual algorithms, any
unexplained or unexpected errors or warnings appearing in the associated log
or on screen output must be described in the documentation for the system
under test. Any warning or error for which this is not the case must be filed
as a software problem report and filed with the DMCCB.

\subsection{Suspension criteria and resumption requirements}
\label{suspension}

Refer to individual test cases where applicable.


\input{specs/specs.tex}
\input{cases/cases.tex}

\appendix

\section{The Hyper Suprime-Cam ``RC'' datasets}

\subsection{RC1}

The original HSC ``release candidate'' dataset was defined as part of testing
release 3.9.0 of the HSC pipeline (derived from the LSST DM Stack). It
consists of 237 visits to the HSC ultra-deep Cosmos field and 83 visits to the
HSC wide survey. Specifically, this includes the following
visits\footnote{Defined using the standard LSST command-line task syntax}:

\subsubsection{Ultra-deep Cosmos}
\label{sec:hscrc1}

\begin{description}

\item[HSC-G]{\hfill \\ 11690..11712:2\^{}29324\^{}29326\^{}29336\^{}29340\^{}29350\^{}29352}
\item[HSC-R]{\hfill \\ 1202..1220:2\^{}23692\^{}23694\^{}23704\^{}23706\^{}23716\^{}23718}
\item[HSC-I]{\hfill \\ 1228..1232:2\^{}1236..1248:2\^{}19658\^{}19660\^{}19662\^{}19680\^{}19682\^{}19684\^{}\\19694\^{}19696\^{}19698\^{}19708\^{}19710\^{}19712\^{}30482..30504:2}
\item[HSC-Y]{\hfill \\ 274..302:2\^{}306..334:2\^{}342..370:2\^{}1858..1862:2\^{}1868..1882:2\^{}11718..11742:2\^{}22602..\\22608:2\^{}22626..22632:2\^{}22642..22648:2\^{}22658..22664:2}
\item[HSC-Z]{\hfill \\ 1166..1194:2\^{}17900..17908:2\^{}17926..17934:2\^{}17944..17952:2\^{}17962\^{}28354..28402:2}
\item[NB0921]{\hfill \\ 23038..23056:2\^{}23594..23606:2\^{}24298..24310:2\^{}25810..25816:2}

\end{description}

\subsubsection{Wide}

\begin{description}

\item[HSC-G]{\hfill \\ 9852\^{}9856\^{}9860\^{}9864\^{}9868\^{}9870\^{}9888\^{}9890\^{}9898\^{}9900\^{}9904\^{}9906\^{}9912\^{}11568\^{}\\11572\^{}11576\^{}11582\^{}11588\^{}11590\^{}11596\^{}11598}
\item[HSC-R]{\hfill \\ 11442\^{}11446\^{}11450\^{}11470\^{}11476\^{}11478\^{}11506\^{}11508\^{}11532\^{}11534}
\item[HSC-I]{\hfill \\ 7300\^{}7304\^{}7308\^{}7318\^{}7322\^{}7338\^{}7340\^{}7344\^{}7348\^{}7358\^{}7360\^{}7374\^{}7384\^{}7386\^{}\\19468\^{}19470\^{}19482\^{}19484\^{}19486}
\item[HSC-Y]{\hfill \\ 6478\^{}6482\^{}6486\^{}6496\^{}6498\^{}6522\^{}6524\^{}6528\^{}6532\^{}6544\^{}6546\^{}6568\^{}13152\^{}13154}
\item[HSC-Z]{\hfill \\ 9708\^{}9712\^{}9716\^{}9724\^{}9726\^{}9730\^{}9732\^{}9736\^{}9740\^{}9750\^{}9752\^{}9764\^{}9772\^{}9774\^{}\\17738\^{}17740\^{}17750\^{}17752\^{}17754}

\end{description}

\end{document}

% Include all the relevant bib files.
% https://lsst-texmf.lsst.io/lsstdoc.html#bibliographies
\bibliography{lsst,lsst-dm,refs_ads,refs,books}

\end{document}
